\documentclass[a5paper]{scrreprt}
\usepackage{amsmath}
\usepackage{amssymb}
\usepackage[utf8]{inputenc}
\usepackage[ngerman]{babel}
\usepackage{siunitx}
\usepackage{graphicx}
\usepackage{hyperref}
\usepackage{float}
\usepackage{placeins}
\usepackage{wrapfig}
\usepackage{subcaption}
\usepackage{esvect}
\usepackage{braket}
\usepackage{dsfont}

\title{Quantenstatistik}
\author{wir}

%Eigene Kommandos hier
\newcommand{\rhopx}{\rho ( p , x )}
\newcommand{\dpdx}{\int dp^{3N} \, dx^{3N} \,}
\newcommand{\erw}[1]{\langle #1 \rangle}
\newcommand{\var}[1]{( #1 - \langle #1 \rangle )}
\newcommand{\deltaa}{\delta ( a - A )}
\newcommand{\deltab}{\delta ( b - B )}
\newcommand{\wkeit}[1]{W_{#1}(\lowercase{#1})}
\newcommand{\ri}{\hat{\vec{r_{i}}}}
\newcommand{\rj}{\hat{\vec{r_{j}}}}
\newcommand{\psirn}{\Psi ( \vec{r_{1}}, \dots \vec{r_{n}})}
\newcommand{\A}{\hat{A}}
\newcommand{\dop}{\hat{\rho}}
\newcommand{\jnorm}{\frac{1}{J}}
\newcommand{\tr}{\, \text{Tr} \,}


\begin{document}
	\begin{titlepage}
  \centering
  {\scshape \LARGE Universität Regensburg \par}
  \vspace{1cm}
  {\scshape\Large Theoretische Physik IV: \par}
  \vspace{1.5cm}
  {\huge\bfseries Quantenstatistik \par}
  \vspace{2cm}
  \includegraphics[width=0.6\textwidth]{/home/jonas/Dokumente/logo_ur/ur-logo-bildmarke-grau.png}\par
  \vfill
  {\large gesetzt von uns\par}
  \vfill
  {\large \today\par}
\end{titlepage}

	\tableofcontents

\input{Kap/kap1.tex}
	\section{Exkurs: Emergenz der Entropie}
	
	\begin{center}
		\includegraphics[width=0.6\textwidth]{Abb/Emergenz.png}
	\end{center}%
	Man stelle sich zwei mit Gas gefüllte Behälter in zwei verschiedenen Laboratorien vor. Beide besitzen -innerhalb einer gewissen Fehler\-toleranz- die gleichen charakteristischen Zustandsgrößen und lassen sich anhand dieser zum Zeitpunkt $t<0$ nicht unterscheiden. Im Zeitintervall $0<t<m$ werden die beiden Behälter nun in unterschiedliche Richtungen gerührt und sind im folgendem $m>t>rel$ voneinander unterscheidbar. Nach einer gewissen Relaxationszeit $rel$ befinden sich die beiden Gase wieder im Gleichgewicht und sind makroskopisch nicht mehr voneinander zu unterscheiden. Die makroskopische Beschreibung hat das Rühren "vergessen". 

	
\chapter{Grundlegende Konzepte der Statistik und der Wahrscheinlichkeitstheorie}

	\paragraph{klassische Mechanik:}
		Mittelwert, Varianz, Verteilungsfunktionen, Korrelationen ect.
	\paragraph{QM:}
		analoge Bildungen.
		
	\section{Klassische Statistik}
	
	\begin{figure}[H]
		\includegraphics[width=0.9\textwidth]{Abb/abb21.png}
		\caption{Links: bildliche Darstellung der Menge aller Ensemble im 6N-Dimensionelen Raum mit Achsen $p$ und $x$.\\ Rechts: Histogramm (oft $\frac{\text{Häufigkeit}}{\text{Anzahl}}$, so dass dieses normiert ist.)}
		\label{HTT}
	\end{figure}
		
	Aus der Abb. \ref{HTT} folgt intuitiv die Definition des Mittels. Um dieses zu erhalten Summiert man alle Werte $A(p^{(j)},x^{(j)})$ des betrachteten Ensembles $\{p^{(j)},x^{[j]}\}, j=1,...,J $ und Teilt durch die Anzahl der Elemente des selbigen.
	\begin{align*}
		\langle A \rangle = \frac{1}{J} \sum_{j=1}^{J} A(p^{(j)},x^{(j)}) \\
	\end{align*}
	
	\noindent Durch den Übergang zu beliebig genauen Messgenauigkeiten, wird die diskrete Verteilung in Abb. \ref{HTT} zu einer kontinuierlichen Kurve. (Übergang zum Kontinuumslimes.)
	
	\begin{figure}[H]
		\includegraphics[width=0.9\textwidth]{Abb/abb22.png}
		\caption{Übergang zum "Kontinuumslimes" mit Wahrscheinlichkeit $W_A(a)$.}
	\end{figure}
	\noindent Wie oben ist auch diese Kurve normiert. Die Häufigkeit wird ersetzt durch $W_A(a)$, welche die Wahrscheinlichkeit angibt, für ein gegebenes Ensemble, eine Messgröße $A$ mit dem Wert $a$ zu Messen. 	
	\newline \newline
	Im Folgenden sollen nun noch kurz einige wichtige Begrifflichkeiten aus der Statistik definiert werden.
	
	\paragraph{Varianz:}
	\[\langle [A - \langle A \rangle]^2 \rangle  = \langle A^2 \rangle - \langle A \rangle^2 \]
	\paragraph{Weitere Momente:}\footnote{Definition und mehr auch im Bronstein "Taschenbuch der Mathematik"}
	\[ \langle A^k \rangle = \frac{1}{J} \sum_{j=1}^{J}(A(p^{(j)},x^{(j)}))^k \]
	\paragraph{Phasenraumdichte:}\footnote{$ \delta(p-p^{(j)}) = \delta(p_{1x}-p_{1x}^{(j)}) \cdot \delta(p_{1y}-p_{1y}^{(j)})\cdot ... \cdot \delta(p_{Nz}-p_{Nz}^{(j)})$.}
	\[ \rho (p,x) = \frac{1}{J} \sum_{j=1}^{J} \delta(p-p^{(j)}) \delta(x-x^{(j)}) \]
	
	\noindent Mithilfe der Phasenraumdichte lässt sich jetzt der Mittelwert auch mithilfe eines Integrals beschreiben: 
	\[ \langle A \rangle \int dp^{3N} dx^{3N} \rho (p,x) A(p,x)\]
	
	\subsection{Motivation für die Einführung der Phasenraumdichte:}
	Zum einen erlaubt die Einführung durch Produktbildung die Trennung der Eigenschaften von Beobachtungsgrößen und Ensemble, zum anderen ist dies die mathematische Umsetzung des Übergangs vom mikroskopischen zum makroskopischen (Coarse-Graining). 
	
	\subsection{Generische Eigenschaften von $\rhopx$}

\begin{enumerate}
	\item $\rhopx$ ist reell\\
	\item $\rhopx \geq 0$\\
	\item $\dpdx \rhopx = 1$
\end{enumerate}
\paragraph{Bemerkung:} Wir können $\rhopx$ als Wahrscheinlichkeitsdichte
interpretieren.
\begin{align*}
	&\erw{A} = \dpdx \rhopx A(p,x)\\
	&\wkeit{A} = \erw{\deltaa} = \dpdx \rhopx \delta ( a - A(p,x) )
\end{align*}

	\subsection{Korrelation}

\begin{center}
	\includegraphics[width=0.8\textwidth]{Abb/korrelation.pdf}
\end{center}

\paragraph{Bemerkung:} Neben Mittelwerten von Observablen sind auch deren
Korrelationen ein wichtiges Charakteristikum eines physikalischen Zustands.\\
$\rightarrow$ Fluktuations-Dissipations-Theorem

\begin{align*}
	K_{AB} &= \erw{\var{A} \var{B}}\\
		   &= \frac{1}{J} \sum_{j=1}^{J} \big( A^{(j)} - \erw{A} \big) \big( 
		      B^{(j)} - \erw{B} \big)\\
		   & A^{(j)} = A(p^{(j)}, x^{(j)})
\end{align*}

\paragraph{Paarwahrscheinlichkeit:} 
\begin{align*}
	\wkeit{AB} &= \erw{ \deltaa \deltab }\\
				   &= \dpdx \rhopx \deltaa \deltab
\end{align*}

\paragraph{Eigenschaften:}
\begin{align*}
	1.\; &\wkeit{A} = \int db \, \wkeit{AB}\\
	     &\wkeit{B} = \int da \, \wkeit{AB}\\
	2.\; &K_{AB} = \int da \, db \, ( a - \erw{A} ) (b - \erw{B} ) \, 
	     \wkeit{AB}
\end{align*}

\paragraph{Definition: statistisch unabhängig}
\[
	\wkeit{AB} = \wkeit{A} \cdot \wkeit{B}
\]
Beispiel für statistisch unabhängige Größen, sind die einzelnen Würfe beim Münzwurf.

\subparagraph{Anwendung:}
Betrachten wir zwei komplett getrennte Gase und wollen deren Eigenschaften
untersuchen, so gilt:
\begin{align*}
	&\rhopx = \rho (p_{1}, x_{1}) \rho (p_{2}, x_{2})\\
	&p=(p_{1},p_{2}) \quad x = (x_{1}, x_{2})
\end{align*}

	\section{Quantenstatistik}
	\subsection{Quanten-Mittelwerte}

\subparagraph{Beispiel:} Fock-Raum als Analogon zum klassischen Phasenraum für $N$
Fermionen\\

Vielteilchenwellenfunktionen von Fermionen sind vollständig antisymmetrisiert
("`Slater-Determinante"'), Bosonen haben dagegen eine vollständig symmetrisierte 
Wellenfunktion

\[
    \hat{H} = \sum_{i=1}^{N} \left( - \frac{1}{2m} \frac{d^{2}}{d\ri} + V_{ex}
               (\ri) \right) + \frac12 \sum_{i \neq j}{N} v(\ri - \rj)
\]

z.B.: $v(\vec{r}) = \frac{R^{2}}{|\vec{r}|}$

\paragraph{Zeitentwicklung:}
\[
    i \partial_{r} \psirn = \hat{H} \psirn
\]

Zum Phasenraum für Fermionen:
Dieser Phasenraum wird aufgespannt durch alle möglichen Slaterdeterminanten, die man
bilden kann: 
\[{}
    \ket{SD_{1}} \ket{SD_{2}} \dots{}
\]
Eine Wellenfunktion wird in diesem Raum durch einen Punkt dargestellt.
Sei nun $\ket{j}$ eine Menge/Ensemble von Mikrozuständen der QM (Wellenfunktionen)
$j = 1,\dots, J$. Die Zustände $\ket{j}$ seien kompatibel mit den experimentellen 
Randbedingungen, z.B. Teilchenzahl, Gesamtenergieinhalt,...\\
Für einen Teilchenzustand $\ket{j}$ ist sein Beitrag zum Ensemble:
\begin{align*}
    &A^{(j)} := \braket{j | \hat{A} | j}\\
    &\erw{\A} := \frac{1}{J} \sum_{j=1}^{J} \braket{j| \A | j} \quad 
    \text{(Mittelwert)}
\end{align*}

    \subsection{Dichteoperator}

\paragraph{Vollständigkeit:} $\mathds{1} = \sum_{\mu} \ket{\mu} \bra{\mu}$, $\mu$ 
ist ein Basisvektor im Vektorraum

\begin{align*}
    \erw{\A}&= \frac{1}{J} \sum_{j=1}^{J} \braket{j | A | j} = \frac{1}{J} 
               \sum_{j} \sum_{\mu} \braket{j|\mu}\braket{\mu|\A|j}\\
    \quad   &= \frac{1}{J} \sum_{j} \sum_{\mu} \braket{\mu|\A|j}\braket{j|\mu}\\
    \dop    &= \jnorm \sum_{j} \ket{j}\bra{j} \quad 
              \text{Dichteoperator der QS}\\
    \erw{\A}&= \sum_{\mu} \braket{\mu| \A \dop |\mu} = \tr (\A \dop)
\end{align*}

\paragraph{Anwendung im mikroskopischen System:}
\[{}
    \dop = \frac12 \left( \ket{\uparrow} \bra{\downarrow} + \ket{\leftarrow} 
                           \bra{\leftarrow} \right){}
\]
{}
Die Zusammensetzung der Ensembles soll in der Übung behandelt werden.

\paragraph{Eigenschaften von $\dop$:}
\begin{enumerate}
    \item Hermitizität: $\braket{\mu|\dop|\nu} = \braket{\nu|\dop|\mu}^{*}$\\
    \item Positivität: $\braket{\dots|\dop|\dots} \geq 0$ für jedes 
          $\ket{\dots}$\\
    \item Normierung: $\sum_{\mu} \braket{\mu|\dop|\mu} = \tr \dop = 1$
\end{enumerate}



\chapter{Mikrokanonische Ensemble}
%\begin{figure}
    %\centering

%\end{figure}
Im mikrokan. Ensemble gehen alle Mikrozustände im kompatiblen Phasenraum mit dem gleichen Gewichtsfaktor ein.
Bedingung für die Energie ist also $E > E_n > E - \Delta E$ ($E_n$ ist Energie von Mikrozustand $\ket n$)
\begin{equation*}
g(E) = \sum_n{}\int_{E-\Delta E}^{E}dE'\delta (E' - E_n)
\end{equation*}
\begin{equation*}
\rho(E_n, E) = \begin{cases}\frac{1}{g(E)} , E>E_n>E-\Delta E\\0 , sonst\end{cases}
\end{equation*}

\section{Eigenschaften der integrierten Zustandsdichte}
\begin{equation*}
g(E) = \sum_n{}\int_{E_0}^{E}dE'\delta (E' - E_n)
\end{equation*}
\begin{equation*}
\rho(E_n, E) = \begin{cases}\frac{1}{g(E)} , E>E_n>E-\Delta E\\0 , sonst\end{cases}
\end{equation*}
(Integral unabhängig von unterer Grenze)
Skalenargument: Gesucht $g(E, N)$, mit $N$ makroskopisch.
\begin{align*}
    g(E, N) &= g(E_1, N_1)g(E_2, N_2)\\
        &= g(E/2, N/2)^2\\
        &\approx g(E/4, N/4)^4\\
    \lim_{N\to\infty}g(E, N) &\approx g_1(E/N, 1)^N
\end{align*}
Mit $g_1$ als intergrierte Zustandsdichte für System mit einem Teilchen.
Bsp.: Box mit einem freien Teilchen (eindimensional)
Abzählen liefert: $g_1(E) \propto e^a$ mit $a = \frac a2$
\begin{align*}
    g(E, N) &\approx [(\frac EN)^a]^N\\
        &= \exp (N\log(\frac EN)^a)\\
        &= \exp (aN\log(\frac EN))
\end{align*}
\paragraph*{Diskussion:}
mittlere Energie von eingeschlossenen Teilchen (kalt):
$\epsilon _k := \frac EN$\\
mittlere Energie von eingeschlossenen Teilchen (warm):
$\epsilon _w = (1+\delta)\epsilon _k$
$$g(E, N) = \exp(aN\log(\bar E))$$

\begin{equation*}
    \frac{g(E_w, N)}{g(E_k, N)}=
    \exp(aN\log(\frac{\epsilon _w}{\epsilon _k}))=
    \exp(aN\log(1+\delta))=\exp(aN\delta)
\end{equation*}
für $\delta$ klein gegen $1$

\chapter{Kanonische Ensemble}
\paragraph*{Motivation:}
\begin{itemize}
    \item Experimente finden nur selten unter voller Isolation statt.
Insbesondere ist ein thermischer/energetischer Austausch mit einer Umgebung für makroskopische Systeme die Regel.
    \item \begin{align*}
            g(E, N) &= \int_{E_0}^{E}dE'\Omega(E, N)\\
            \Omega(E, N) &= \sum_n\delta(E-E_n)\\
            \rho(E_n, E) &= \begin{cases}\frac{1}{g(E)} , E>E_n>E-\Delta E\\0 , sonst\end{cases}
        \end{align*}
\end{itemize}

\section{Der reduzierte Dichteoperator eines mikrokanonischen Ensembles}
In einem getrennten Wärmebad mit festem E, N (2 verschiedene Temperaturen) seien:
\begin{itemize}
    \item $\ket{1m}$ Eigenzustand von $H_1$
    \item $\ket{2n}$ Eigenzustand von $H_2$
\end{itemize}
$\ket{mn}=\ket{1m}\otimes\ket{2n}$\\
Energie: $E_{mn} \approx E_{1m} + E_{2n}$
\begin{equation*}
    \rho _{12}(E_{mn}) = \begin{cases}\frac{1}{g(E)} , E>E_{mn}>E-\Delta E\\0 , sonst\end{cases}
\end{equation*}
Mit $\rho _{12}$, der Wahrscheinlichkeit, den spezifischen Zustand $mn$ im Ensemble (mikrokan.) von $S_1 + S_2$ anzutreffen, falls die Gesamtenergie E ist.
Gesucht wird nun $\rho _1 (E_{1m})$,
das heißt die Wahrscheinlichkeit in $S_1$ den Mikrozustand $\ket{1m}$ anzutreffen, wenn die Energie des Gesamtsystems von $S_1+S_2$ auf $E$ fixiert ist.\\
Wir wissen:\\
$E - \Delta < E_{1m} + E_{2n} < E$\\
Falls in $S_1$ die Energie $E_{1m}$ vorliegt, d.h. $S_1$ ist im Zustand $\ket{1m}$, dann gilt:\\
$E - \Delta -E_{1m} < E_{2n} < E - E_{1m}$
Nun gibt es $g_2(E-E_{1m})$ Zustände in $S_2$, die mit dieser Bedingung kompatibel sind.
Das heißt, dass ein Bruchteil der Zustände des Gesamtsystems:\\
$\frac{g_2(E-E_{1m})}{g(E)}$
es zulässt, dass $S_1$ im Zustand $\ket{1m}$ ist. Wir schließen:
\begin{equation*}
    \rho(E_{1m}) = \frac{g_2(E-E_{1m})}{g(E)}
\end{equation*}
\section{Energie der Temperatur}
$E_1 = \braket{H_1}; E_2 = \braket{H_2}; E = E_1 + E_2$
$$\rho(E_{1m}) = \frac{g_2(E_2+E_1-E_{1m})}{g(E)}$$
Was beschränkt $E_1 - E_{1m}$? (enspricht der Schwankung in $S_1$)\\
Wir haben sicher: $|E_1 - E_{1m}| \leq \Delta _2$
\begin{equation*}
    \frac{|E_1 - E_m|}{E_2}\leq\frac{\Delta E_2}{E_2}=\sqrt{\frac{\Delta E_2}{E_2^2}}\approx\frac 1{\sqrt{N_2}}
\end{equation*}
Wenn nun $S_2$ tatsächlich im ’Bad’,
also makroskopisch, dann ist\\ $\frac 1{\sqrt{N_2}} << 1$.\\
$g_2$ wächst schnell, deshalb entwickle $\log(g_2)$
\begin{equation*}
    \log(g_2(E_2+E_1-E_{1m})) = \log(g_2(E_2))+\beta(E_1-E_{1m}) + \mathcal{O}{(N^0)} + \mathcal{O}{(N^{-1})}
\end{equation*}
$$\beta = \frac{\partial \log(g_2(E_2))}{\partial E_2}$$
\paragraph*{Diskussion:}
\begin{enumerate}
    \item Term: $\log(g_2(E_2))\propto N_2$
    \item Term: $\frac{\partial \log(g_2(E_2))}{\partial E_2}\propto N_2\frac{\partial \log(E_2/N_2)}{\partial E_2}\propto \frac{N_2}{E_2}$ (Energie pro Teilchen)
\end{enumerate}
Einsetzten liefert:
\begin{align*}
    \rho_1(E, m)&=\frac{g_2(E_2)}{g(E)}\exp(\beta(E_1-E_{1m}))\\
            &=\frac{g_2(E_2)}{g(E)}\exp(\beta(E-E_2))\exp(-\beta E_{1m})\\
            &=\frac{g_2(E_2)}{g(E)}\frac{\exp(-\beta E_2)}{\exp(-\beta E)}\exp(-\beta E_{1m})
\end{align*}
$\frac{g_2(E_2)}{g(E)}\frac{\exp(-\beta E_2)}{\exp(-\beta E)}$ entspricht der Systeminformation des Bads\\
$\exp(-\beta E_{1m})$ entspricht der relativen Gewichtsfunktion verschiedener Zustände\\
Normierungsbedingung:$$\sum_{m\in S_1}\rho_1(E, m) \approxeq 1$$
Aus dieser Bedingung lässt sich der Vorfaktor berechnen:
\begin{align*}
    \rho_1(E, m)&=\frac 1Z\exp(-\beta E_{1m})\\
    Z&:=\sum_{m\in S_1}\exp(-\beta E_{1m})
\end{align*}
\subparagraph*{Bemerkung:}
\begin{align*}
    \beta &= \frac 1{k_BT} & T &:= Temperatur\\
          &     & k_B &:= Boltzmannfaktor
\end{align*}
\paragraph*{Spezialfall:}
\begin{align*}
    H(p, x)&= T(p) + W(x)\\
    \rho(p, x)&=\frac 1Z\exp(-\beta T(p))\exp(-\beta w(x))
\end{align*}
Dann lautet die Normierungsbedingung:
$$\frac1Z\int dp\exp(-pT(p))\int dx\exp(-\beta H(x))=1$$
Daraus folgt: $$z=z_pz_x$$
Teilchen in höheren Dimensionen:
\begin{align*}
    \rho(\vec p\vec x))&=\frac1Z\exp(-\beta H(\vec p\vec x)))\\
    Z&=\int \frac{d\vec p\vec x)}{(2\pi\hbar)^d}\exp(-\beta H(\vec p\vec x)))
\end{align*}


\end{document}

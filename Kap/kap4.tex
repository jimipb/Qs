\chapter{Mikrokanonische Ensemble}
%\begin{figure}
    %\centering

%\end{figure}
Im mikrokan. Ensemble gehen alle Mikrozustände im kompatiblen Phasenraum mit dem gleichen Gewichtsfaktor ein.
Bedingung für die Energie ist also $E > E_n > E - \Delta E$ ($E_n$ ist Energie von Mikrozustand $\ket n$)
\begin{equation*}
g(E) = \sum_n{}\int_{E-\Delta E}^{E}dE'\delta (E' - E_n)
\end{equation*}
\begin{equation*}
\rho(E_n, E) = \begin{cases}\frac{1}{g(E)} , E>E_n>E-\Delta E\\0 , sonst\end{cases}
\end{equation*}

\section{Eigenschaften der integrierten Zustandsdichte}
\begin{equation*}
g(E) = \sum_n{}\int_{E_0}^{E}dE'\delta (E' - E_n)
\end{equation*}
\begin{equation*}
\rho(E_n, E) = \begin{cases}\frac{1}{g(E)} , E>E_n>E-\Delta E\\0 , sonst\end{cases}
\end{equation*}
(Integral unabhängig von unterer Grenze)
Skalenargument: Gesucht $g(E, N)$, mit $N$ makroskopisch.
\begin{align*}
    g(E, N) &= g(E_1, N_1)g(E_2, N_2)\\
        &= g(E/2, N/2)^2\\
        &\approx g(E/4, N/4)^4\\
    \lim_{N\to\infty}g(E, N) &\approx g_1(E/N, 1)^N
\end{align*}
Mit $g_1$ als intergrierte Zustandsdichte für System mit einem Teilchen.
Bsp.: Box mit einem freien Teilchen (eindimensional)
Abzählen liefert: $g_1(E) \propto e^a$ mit $a = \frac a2$
\begin{align*}
    g(E, N) &\approx [(\frac EN)^a]^N\\
        &= \exp (N\log(\frac EN)^a)\\
        &= \exp (aN\log(\frac EN))
\end{align*}
\paragraph*{Diskussion:}
mittlere Energie von eingeschlossenen Teilchen (kalt):
$\epsilon _k := \frac EN$\\
mittlere Energie von eingeschlossenen Teilchen (warm):
$\epsilon _w = (1+\delta)\epsilon _k$
$$g(E, N) = \exp(aN\log(\bar E))$$

\begin{equation*}
    \frac{g(E_w, N)}{g(E_k, N)}=
    \exp(aN\log(\frac{\epsilon _w}{\epsilon _k}))=
    \exp(aN\log(1+\delta))=\exp(aN\delta)
\end{equation*}
für $\delta$ klein gegen $1$

